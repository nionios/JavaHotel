\section{Ανάλυση Μεθόδου}
Το πρόγραμμα λειτουργεί με 4 βασικές οντώτητες:
\begin{itemize}
    \item Τον \textbf{HRClient}. Χρησιμοποιήται από τον τελικό
        χρήστη. Εδώ ο χρήστης εισάγει τα δεδομένα που είναι απαραίτητα, δηλαδή
        της πληροφορίες όπως την εντολή που θέλει να τρέξει, το hostname του
        υπολογιστή του, τον τύπο του δωματίου το οποίο θέλει να ακυρώσει ή να
        κρατήσει, το όνομά του.
        \\
        Ο HRClient με την σειρά του μεταβιβάζει τα
        δεδομένα αυτά στον HRServer ο οποίος επιστρέφει την κατάλληλη απάντηση
        έπειτα από την επεξεργασία αυτών των δεδομένων.
        \\
        Στην περίπτωση που ο χρήστης επιθυμεί να κλείσει δωμάτια τα οποία έχουν
        εξαντληθεί, ερωτείται για το αν θέλει να ειδοποιηθεί όταν τα εν λόγω
        δωμάτια γίνουν ξανά κενά. Τότε, ο HRClient και ο HRServer "ανταλλάζουν
        ρόλους" και ο HRServer περιμένει να αδειάσει ένα δωμάτιο για να κάνει
        RMI Callback στον HRClient, ειδοποιώντας τον για την ευκαιρία κράτησης.
    \item Τα ενδιάμεσα \textbf{interfaces}. Ουσιαστικά\
        μεσολαβούν στην επικοινωνία των HRClient και HRServer με το να
        συμπεριφέρονται σαν virtual μέθοδοι (C++) ή σαν header files (C,C++)για
        τον HRClient στην περίπτωση του HR.java και για τον HRServer στην
        περίπτωση του EmptyRoomListener.java
        \\
        Δηλαδή, περιέχουν το γενικότερο περίγραμμα των κλάσεων HR και
        EmptyRoomListener χωρίς να περιγράφουν την λειτουργία τους.
        Αυτό επιτρέπει στον HRClient και HRServer να χρησιμοποιούν αναφορές
        στις κλάσεις αυτές στον πηγαίο τους κώδικα χωρίς να γνωρίζουν την
        υλοποίηση (implementation) των κλάσεων αυτών.
        \\
        Η ηλοποίηση των κλάσεων αυτών γίνεται στα αρχεία HR\_Impl.java για το
        HR.java και το HRClient.java για το EmptyRoomListener.java
    \item Το \textbf{RMI Registry}. 
        Μεσολαβεί στην επικοινωνία των HRClient και HRServer σε χαμηλότερο
        επίπεδο, πρακτικά "γεφυρόνοντας" τις δύο οντώτητες μέσω δικτύου.
    \item Τον \textbf{HRServer}.
        Εκτελεί τους απαραίτητους ελέγχους στα δεδομένα που του μεταβιβάζονται
        από τον HRClient και τα επεξεργάζεται έτσι ώστε να τα μετατρέψει σε
        χρήσιμη μορφή πληροφορίας (objects, lists, κτλ).
        \\
        Στην συνέχεια επιστρέφει τις κατάλληλες απαντήσεις στον HRClient, ο
        οποίος τις ερμηνεύει για να καταλάβει τι συνέβη με τα δεδομένα που
        έστειλε.
\end{itemize}
\begin{figure}[ht]
    \centering
    \begin{tikzpicture}
        \node (left) at (0,2.5) {HRClient};
        \node (middle) at (6,5) {RMI Registry};
        \node (middledown) at (6,0) {Interfaces};
        \node (right) at (12,2.5) {RPC Server};
        \draw[->,orange] (left.north)     .. controls +(up:2cm)    and +(left:1cm)    .. node[above,sloped] {\small Επιλογές, Πληροφορίες Κράτησης} (middle.west);
        \draw[<-<,orange] (right.north)   .. controls +(up:2cm)    and +(right:0cm)    .. node[above,sloped] {\small Επιλογές, Πληροφορίες Κράτησης} (middle.east);
        \draw[-<,blue] (middle.south)     .. controls +(down:2cm)  and +(left:1.2cm) .. node[above,sloped] {\small Ειδοποίηση} (right.west);
        \draw[-<,red] (middle.south)      .. controls +(down:5cm)  and +(left:1.5cm) .. node[below,sloped] {\small Απάντηση} (right.west);
        \draw[->,blue] (middle.south)     .. controls +(down:2cm)  and +(right:1.2cm) .. node[above,sloped] {\small Ειδοποίηση} (left.east);
        \draw[->,red] (middle.south)      .. controls +(down:5cm)  and +(right:1.5cm) .. node[below,sloped] {\small Απάντηση} (left.east);
        \draw[-,brown] (left.south)  .. controls +(down:2cm) and +(left:0cm)   .. node[below,sloped] {\small Περίγραμμα Κλάσης} (middledown.west);
        \draw[-,brown] (right.south)  .. controls +(down:2cm) and +(right:0cm)   .. node[below,sloped] {\small Περίγραμμα Κλάσης} (middledown.east);
    \end{tikzpicture}
    \caption{\footnotesize{Η αρχιτεκτονική του προγράμματος}}
    \label{fig:searx-oper}
\end{figure}
Παραπάνω αρχεία που βρίσκονται στους φακέλους με τον πηγαίο κώδικα είναι
είτε βοηθητικές συναρτήσεις είτε μέρος αυτών των 4 οντωτήτων αλλά χωρισμένες
σε διαφορετικά αρχεία για ευκολία χειρισμού και κατανόησης.
\\
Έχει γίνει προσπάθεια να γίνει πρόληψη για τα σφάλματα του χρήστη, για
παράδειγμα η εσφαλμένη είσοδος δεδομένων με διαφορετικό τύπο από τον ζητούμενο
κ.α.
\section{Ενδεικτικά Τρεξίματα}
